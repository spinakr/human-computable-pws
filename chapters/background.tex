\chapter{Background}\label{chp:background}

\section{Authentication}

\subsection{Passwords}
Passwords are the common way of authenticating users upon access to sites on the Internet, the idea is that only the user and the target service knows the password, and the user have to provide the correct password before access are granted. Passwords are a much discussed theme and claiming that passwords are usually not used in the correct manner is not an overreaction. The main problem seems to be the fact that good passwords and the human memory does not go well together. For passwords to be sufficient as authentication each user has to be forced into using long complex password, or even use one generated for them, with the problem being that it is easily forgotten. Further more, if a user was able to memorize \emph{one} "good" password, he will probably use this for all of his accounts, so that if one of the services is compromised and user information leaked, all his accounts may be compromised. With all of this in mind it is easy to say that everybody should use complex, unique passwords for each account, but in practice this is not feasible.
\par Password authentication requires the authenticating server to store something related to the password, if this is stolen the password will in most cases be compromised as well, even if the server did not store the clear text password, attackers will, in most cases, be able to retrieve the password eventually. After obtaining the username and password for one service the attacker would try this user data on other services and compromise these as well. 

Miller \cite{magic-seven_miller} showed that the human brain cannot store more than $7\pm 2$ chunks of information in immediate memory.
\subsection{Attacks}
\subsection{Password Storage}
active and passive attacks  
\subsection{Alternative authentication methods}

\section{Human Behavior}

\section{Password Management}

\section{Usability and Security Challenges}

\section{Mnemonic Techniques}

\section{Browser Extensions}

\section{Usability Model}

\section{Security Model}




