\chapter{Usability Experiment}\label{ch:experiment}
This chapter will present the design and execution of an experiment trying to measure  the performance of the human computable password management scheme. The experiment is design as a web application implementing the scheme allowing participants to test how the scheme would work in practice while measuring how fast the computation is performed. The application thus acts as both a demonstration app and a tool for gathering performance data. It has four sections designed to help the participants understand the scheme and get familiar with the computation technique. First as demonstration video is displayed explaining how to compute a password from a challenge, then the user is asked to enter some demographic data. Next is the practice section, where the user is supposed to practice doing the calculation until being able to solve challenges without error. Finally is the experiment part, which is timed and correctness monitored. After finishing the calculations the user will submit the data and can chose to continue doing more experiments. 
\section{Experiment Objective.}
\todo[inline]{Explain what the goal of the experiment is. What results I'm looking for/what the purpose is.}
\todo[inline]{NOTE: The experiment application is not only recording data, it works as a presentation of the scheme, trying to make it easy to understand so that most people should be able to use it. }
The goal of the experiment is to measure how hard it is for a user to learn the scheme, and namely how fast users do the calculations. The experiment will thus measure the calculation times of different users, and also monitor their progress after several trials. Hopefully a user will be able to calculate passwords without to many mistakes. The failure rate is maybe the most important variable to measure; if a user average more than one mistake for each password calculated the scheme would not work in practice, since most passwords calculated would be wrong. 
\par It is important to note that this project does not see the human computable password management scheme as a replacement for the widen used standard password managers. It is regarded as a solution for users interested in a secure and reliable way of keeping track of strong password, without having to trust a password management service or application. This experiment will thus test if it is even possible, for a user willing to go through the trouble of learning a secret mapping, to use the scheme as an everyday solution.

\par The objectives of the experiment can be sumarized as the following.
\begin{itemize}
    \item Measure the average calculation times, both for each participants and for the whole population.
    \item Measure how much the users improve after several trials.
    \item Measure the average failure rate for each participant.
\end{itemize}

\subsection{Method}
The experiment does not try to test any preset hypothesis as it is not clear what to expect regarding either of the tested values. It is not known how hard it is to to the calculations, neither regarding calculation time or failure rates. The results of the experiment will not be a clear answer to any hypothesis, but rather a basis for further data collection through larger scale experiments, in exmple using crowd sourcing or social medias.

\par Exploratory data analysis (EDA) was introduced by John W. Turkey~\cite{turkey} which involves analysis of data without a prior hypothesis to test. The technique promotes exploration of data to possible find characteristics not previously considered, and essentially suggesting hypotheses to be tested later surveys or experiments. Velleman and Hoaglin~\cite{exploratory-analysis} describes EDA as a contrast to the formal scientific method involving stating a hypothesis, collecting data and applying a statistical test of the hypothesis. EDA often involves making graphical representations of the data, and then try to find interesting characteristics and relations. EDA does not conclude with a hypothesis test based on the collected data, but is the first step of an iterative process trying to reveal facts. 
\par This project does not intend to test any hypothesis, but rather gather data thought to be relevant, in this case, calculation time and failures, then display the data as graph and investigate relationships between different statistics. The findings of the experiment should be seen only as a prerequisite for later experiments, the patterns found can be used to form a more specific statistical test.
\par The project will not store any personal information about the participants, and should thus not be subject to notification to the "Norwegian Social Science Data Service"\footnote{Is my research project subject to notification? - \url{http://www.nsd.uib.no/nsd/english/pvo.html}}. The experiment will only record an anonymous id plus the experiment results consisting of computation time and correctness of the calculations done. Some demographic data will also be recorded (age, area of study), but the data can not be connected to person and are thus not regarded as personal information.


\section{Experiment Setup}
\todo[inline]{What data will be recorded, choices regarding the secret mapping etc.}
The experiment will present and demonstrate the calculation technique to the user, which then will be given a chance to practice until fairly familiar with the mechanics. Finally the user will be asked to calculate a complete password challenge, from which the time spent on each single digit challenge will be recorded, as well as if the calculation was correct or not. The practice section will allow the user to try and fail, using backspace to go back and forth between challenges while also giving feedback on the correctness. The experiment view on the other hand will not give any feedback and does not allow the user to go back after entering a character. This is done to make sure all mistakes are recorded, even if it is only a missclick. 
\subsubsection{Secret mapping} The biggest decision made in regards to the experiment design was how to simulate "recalling" from memory. It would not be feasible to ask all the participants to memorize a secret mapping on beforehand as this would probably make it impossible to find volunteers. The chosen solution to this problem is to include a "cheat sheet" in addition to the displayed challenges, this will be a list of object to digit mappings shown separately but in the same view as the challenge.
\par After some testing it became apparent that the there was a big difference between recalling from memory and actually "reading" from a list, which this approach eventually is. To make the operation more similar to the desired recall operation, the mapping was changed from random to mirror the alphabet with accompanying position(e.g A=1, B=2, C=3). This way one does not have to read in the table to know correct mapping for all letters at least. Most users will be able to know instantly what the mapping for at least the first and eventually, after some practice, all the letters. The is the closest way of mimicking the actual operations of the scheme without having the participants memorize an actual mapping. It is not in any way certain that this shortcut reflect the real world act of recalling from memory, but it is assumed to be "close enough" for the experiment. The author of this project did memorize a mapping of both 10 and later 20 mappings without significantly different calculation times compared to using the alphabet positions.

\par Section \ref{computation-time} discussed how a single digit challenge can be presented to the user in a logical way, possibly making it more efficient to calculate. The experiment use a similar layout to the one shown in \autoref{challenges} and also as in the Chrome extension in Chapter \ref{app}. 



\subsection{Web Application}
\todo[inline]{Design and architecture of the web application}
\todo[inline]{Discuss if a similar application could be used as an actual password manager. What are the advantages/disadvantages versus the extension implementation. }
The experiment application will be similar in design to the Chrome extension presented in \autoref{app}, but implemented as a web application. The application will consist of four sections in the form of different views in a form. First the user will be presented with a presentation of the scheme and how to calculate passwords.


\subsection{Results}
\todo[inline]{Presentation of the results gather through the experiment (calculation time, failure rate etc. )}
\todo[inline]{Different plots of the data. Not sure what and how to plot the different data??}

\subsection{Discussion}
\todo[inline]{Discuss how the results fit in the usability model, plots $\hat t$ for the different participants}
\todo[inline]{What use cases would require a faster or slower calculation time?}
\todo[inline]{}
