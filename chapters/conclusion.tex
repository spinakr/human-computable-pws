\chapter{Concluding Remarks and Further Work}
This project has presented the human computable password management scheme by Blocki et al.~\cite{hcp-blocki}, as well as the design and construction of two applications implementing the scheme with different purposes. How different characteristic and parameters affect the usability of the scheme was discussed, as well as how password length and number of secret mappings affect the security. Failure rate was introduced as a important factor affecting the usability of the scheme. 
\par The first application is a Google Chrome browser extension, making it easy for users to employ the scheme without too much trouble. It is a fact that the scheme is quite complex and require a lot dedication from the user for it to work. It is thus important to have a tool minimizing extra work required, including management and generation of challenges. Without a tool to take care of these obstacles, the scheme would be very hard to use in practice. With the application created in this project, a user only have to worry about memorizing and rehearsing the secret mapping, all the overhead related to generation, storing and fetching of challenges is handled by the extension. The one choice the user have to make is what category to put the site in, either low, medium or high sensitivity. This measure was introduced to increase the usability by lowering the average number of single digit challenges, while still keeping the important accounts secured.
\par The application is available in the browser, making it very accessible in situations where passwords are needed. The extension monitors the active page browsed by the user and fetches information from DOM allowing it to display the correct challenge at all times, both in regards to which page the user is visiting and how many characters are entered in the password field. E.g. if the user visit google.com the extension will get notice about this and bring up the challenges associated with google.com, if the site is note part of the system, the user will get the chance to add it. When calculating passwords, the challenge on display update for each entered character in the password field.
\par The second application is a web application functioning as a demonstration platform as well as a experiment, gathering usage data. The experiment gathers data related to calculation time and failure rates. It was created to investigate how the scheme performs, with usability characteristics in focus. The participants in the experiment, gets a short introduction and the chance to practice until fairly familiar with the scheme. And are then asked to calculate a complete password consisting of 10 singlet digit challenges, while getting timed.
\par The experiment measured a average calculation time of $10.922$ seconds, with a standard deviation of $3.64$. This is considered to be reasonably good, as it should not limit the usability of the scheme severely. It could also be observed that users improved over time so that average calculation time would probably be even lower after some practice. 
\par A more unanticipated result was the failure rate, which was measured to be $0.058$, meaning that every 17 single digit challenge would be calculated wrong. This result should not be interpreted as a dismissal of the scheme, since different users will achieve different failure rates, and many will be able to calculate correctly most of the time. What should be emphasized though, is that the failure should be investigated closer, to verify that a general user average a low enough failure rate. 


\section{Further Work}

\subsection{Additional Applications.}
\todo[inline]{web application etc...}
This project showed how browser extensions can be used to create an easy to used password management application. This choice was made with a goal of making a quite complex scheme usable without too much overhead and practice. To make the scheme even more accessible, it would be by creating an accompanying web application and possibly a mobile application. These applications could use the same storage, while still syncing the to chrome storage for the browser extension and to mobile storage for the mobile application. It would be useful to have a user management page to overview user data and challenges. The construction presented here does not allow the user to manually configure the account which some users might dislike. 

\subsection{Larger scale experiments}
It is, as mentioned, clear that the failure rate of the scheme is important for it to function in practice. The experiment conducted in this project did not aim to tell a truth or verify a hypothesis, which should be the next step. A typical setup would could be similar to the one presented here, but in a much larger scale, possibly using a crowd sourcing service or social medias, asking the participants to calculate challenges. It would be important to get more samples from each participant, making it easier to calculate different averages after more practice.  
\par It would also be interesting to conduct a survey gathering data about how users rate their account. Typically asking how many they regard as "highly sensitive", "don't-care" etc. This would make it possible calculate an average password length which again could be used to measure failure rates and calculation times more thorough. 
