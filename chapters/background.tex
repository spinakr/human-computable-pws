\chapter{Background}\label{chp:background}


\section{Passwords}
Passwords are the common way of authenticating users upon access to sites on the Internet. The idea is that only the user and the target service know the password, and the user has to provide the correct password before access is granted. Passwords are a much discussed theme and claiming that passwords are usually not used in the correct manner is not an overreaction. The main problem seems to be that good passwords and the human memory does not go well together. For passwords to be sufficient as authentication each user has to be forced into using long complex password, or even use one generated for them, with the problem being that it is easily forgotten. Furthermore, if a user was able to memorize \emph{one} "good" password, he will probably use this for all of his accounts, so that if one of the services is compromised and user information leaked, all his accounts may be compromised. With all of this in mind it is easy to say that everybody should use complex, unique passwords for each account, but in practice this is not feasible. Florêncio and Herley ~\cite{password-habits} conducted a large scale study of passwords habits in 2007, revealing that a user on average has 25 different accounts protected by passwords. On average these sites are protected by about 7 distinct password, where 5 of them are rapidly re-used.
\par Password authentication requires the authenticating server to store something related to the password, if this is stolen the password will in most cases be compromised as well, even if the server did not store the clear text password. Attackers will, in most cases, be able to retrieve the password eventually. After obtaining the username and password for one service the attacker would try this user data on other services and compromise these as well. 
\par If a user was to have different passwords for each site, these might still easily be compromised. Ives et al. ~\cite{domino-effect} discuss this "domino effect", where intrusion to one domain can compromise several others, if users have re-used passwords.  A normal user will typically try to log in by trial-and-error ~\cite{single-pw-auth}, if the first password does not work, the user will try with another of his passwords. This way may passwords might be lost through phishing attacks where a user is tricked into trying to log in to a fake site. It is thus clear that some kind of system is needed to allow a human user to manage strong passwords. The best case would be if each user, for each of his 25 accounts had a unique password of satisfying strength, this is of course not possible.

\subsection{Password strength}\label{password-strength}
How to measure the strength of passwords is a well known and discussed problem, but the naive approach says that password strength is related to how strong a password is against brute force attacks \cite{password-strength}. Length and complexity are the most thought of parameters to measure such strength. A perfect password would thus be one as long as allowed by the system consisting of random characters from all possible characters, this one would also be changed frequently. All these characteristics challenge how the human brain works. In addition to the objective strength of the password, techniques making it harder for a computer to repeatedly try different passwords may be applied. Such techniques include CAPTACHAS \cite{captcha} which are puzzles supposed to require a human to be able to solve, thus making brute force using a computer harder. Or techniques making it easier for the user to remember multiple strong passwords.
\par Yan et al.~\cite{memorability_yan} investigate the trade-off between security of passwords versus memorability allowing humans to remember them. An important regarding this trade-off is that most sites applying advice and policies on how to create strong passwords, do not take into account if the recommended passwords are hard or easy to remember. There is no point in having a strong password if the user is going to forget it. Yan et al. suggest that passwords should appear random but be constructed using a  mnemonic structure such ass passphrases. The idea here is to generate a random looking password by memorizing a familiar sentence and using the first letters of each word as the password. 
\par Florêncio et al.~\cite{strong-pws_florencio} investigate another matter; do strong passwords accomplish anything? The point is that no matter how long and complex password users chose they are still subject to the most dangerous and common threats (phishing, keylogging or access attacks), as discussed in the previous section. The reason for enforcing strong passwords seems to be to protect against bulk guessing attacks, against other attacks, typically offline attacks, shorter passwords is usually sufficient. 

\paragraph{Password strength meters} are a common way used by many sites on the Internet to aid their users when selecting passwords. Common meters use colored progression bars together with a word or short comment stating if the entered password is evaluated as "bad", "poor" or "strong". Ur et al. \cite{password-meters2} found that password meters actually lead users into choosing longer and stronger passwords, but they also argue that enforcing such policies might frustrate users and possibly lead them into writing passwords down, use weak password management schemes or re-use the same password across many sites. The most common requirements used by passwords meters of known web services can be summarized as the following \cite{password-meters}:
\begin{itemize}
    \item{Length and character selection} are part of most password meters. It is normal to disallow passwords shorter, and sometimes longer, than a given range, which may vary. A variation of different characters can be required, namely different kinds of symbols or capital letters. Spaces may be allowed in between other characters, at the start or end of the password, or not at all. Some sites checks for sequences of the same character as well.
    \item{Personal information.} Information registered by the user are evaluated by some meters, typically name, email and date of birth are checked against in original and transformed forms. This means that a password like "4nD3r?1991" ("anders1991" transformed) which look strong, will be evaluated as weak.
    \item{Dictionary checks.} Make sure that the password does not include any dictionary words by matching it with a dictionary of common words. 
\end{itemize}

\paragraph{Entropy}
\todo{write about entropy, \cite{password-policy, shannon-entropy}}


\par The conclusion on what "good" passwords are, is not clear, but the one thing agreed upon seems to be that re-use of passwords are the biggest threat. It is a fact that the human brain is not capable of remembering different passwords for each account on the Internet, thus the need for an aiding application such as the one discussed in this project. 



\subsection{Attacks}
Passwords are often the only barrier stopping adversaries from directly accessing the accounts of a user. The combination of user name and password are the easiest point of entry to access, and thus the first logical point of attack. There are several methods used to attack password authentication, trying to retrieve passwords. The most important attacks and their respective mitigation technique \cite{nist-guide, strong-pws_florencio}, will be discussed in the following section. 
\paragraph{Capturing} of passwords directly from the server responsible for the authentication involves the attack acquiring password data through breaking into the data storage, eavesdropping on communication channels or through monitoring the user by other means. The first, most basic threat is to simply steal the stored password from an insecure server, this would require a weak configured server storing the passwords in plain text. This is mitigated by storing only cryptographic hashes of passwords, which allows the server to authenticate users while preventing attackers from determining the actual passwords without \emph{cracking} the hashes.

\paragraph{Cracking} requires the attacker to go through several steps. First acquiring the hash of a user account or a whole file of hashes for a site. Next, one would try to find a sequence of strings yielding the same hash as the actual password. How hard it is to crack a hash depends on the strength of the password and can be mitigated by choosing strong passwords and changing these frequently.\emph{ Rainbow tables }~\cite{rainbow-tables} are a technique employed by attackers to speed up this process. Rainbow tables are precomputed table of hashes, allows the attacker to compute a set of hashes once and use these values several times, thus providing a space-time trade-off. This involves using more space, since all the computed hash values would have to be stored somewhere, but allowing a much shorter computation time to brute-force a hash. The technique stores chains of hashes as shown in figure ~\ref{rainbow-table}, storing only the first and last value of the chain. The attacker then searches for a given hash in the set of end points, if no match is found the hash function is applied and a new search conducted. This process continues, until a match is found, the plain text is then computed from the start value of the chain, applying the hash function the same amount of times it took to find a match in the chain. \todo{rewrite - more precise }



\begin{figure}[h]
    \includegraphics[width=\textwidth]{rainbow-table}
    \caption{Rainbow table}
    \label{rainbow-table}
\end{figure}

\subsection{Password Storage}\label{pw-storage}
\todo[inline]{hashing, salt etc..}

\subsection{Alternative authentication methods}
\todo[inline]{alternative auth: graphical passface etc.}


\section{The User}\label{sec:the-user}
"Good passwords" as discussed in \ref{password-strength} does not go well with the human memory. The first limitation which will be an important property later are the limitation to how much data we can store in immediate memory, this limit was showed to be 7 chunks of data at once~\cite{magic-seven_miller}. This data can not be from a random selection which is what a good password requires.   
\todo[inline]{in progress: human behaviour}





\section{Password Management}
As seen in the previous sections passwords introduce a dilemma as passwords are supposed to be hard to "guess" and thus hard to remember. The naive solution to this problem is to either use one password for many accounts or to write down passwords, even tough most users understand that this solution is bad, users will still do it if the password policy required is too complex. To make the process of managing passwords easier several techniques and tools have been suggested. The main techniques used to help manage passwords online are \cite{management-strategies}: 
\begin{itemize}
    \item{Password managment schemes \cite{password-schemes}.} Password creation and memorization techniques assists the user in remembering passwords, trying to circumvent the limitations of the human memory \autoref{sec:the-user}. One way to keep apparently strong unique passwords for all services is to use a base password and apply function using unique information from the different services \footnote{How to Update Your Insecure Passwords and Make Them Easy to Use.\url{http://lifehacker.com/5631203/how-to-update-your-insecure-passwords-and-make-them-easy-to-use}}. The base can be any seemingly random word which is easy to remember, best practice would be to take a sentence or phrase and use the first letters of each word. “May the force be with you.” would yield "mtfbwy" as base, then a rule could be to add the first and last two letters of the service name. With this rule the password for facebook.com would be famtfbwyok.
    \item{Password manager software.} Applications meant to keep passwords safe for the user. These applications can either be stand alone programs or, more common, browser extensions such as LastPass \footnote{LastPass \url{https://lastpass.com/}}. LastPass provides an user interface to generate and store passwords for online services, as well as form fillers to enter them when logging in. The passwords are encrypted using a master password protecting the user credentials against both server leakage and insiders accessing the data.
    \item{Built-in browser password managers.} Most modern browsers provide "remember password" functions. These functions act similar to software like LastPass, by storing the users passwords in some fashion, then reproduce it when login in. 
\end{itemize}

The two latter may seem the easiest to use but requires the user to trust that the implemented systems are secure enough to prevent adversaries, both insiders and outsiders, stealing credentials stored by the services. Software aiding the user by storing passwords often also provide autofill-functions, automatically filling in the username and password of the associated site. Silver et al. \cite{pw-managment-attacks} propose attacks exploiting these autofill functions to extract passwords from the password manager, the most basic example attack is shown in \autoref{rouge-wifi}. Despite of the obvious weaknesses in many password managers, they still argue that password managers can strengthen credential security if implemented correctly.  
\begin{example}\label{rouge-wifi}
    Consider a user connecting to a open wifi at a coffee bar or another public place. It is not unusual to present the user with a "landing page" asking for approval of some usage agreement. The rouge wifi provider could include multiple iframes\footnote{\url{http://www.w3schools.com/tags/tag_iframe.asp}} pointing to the login pages of common sites the user probably have stored credentials for. See \autoref{rouge-wifi-page}. By injecting javascript in the iframes the attacker can extract all the usernames and passwords autofilled by the password manager. Note that the iframes displayed in the figure are not visible to the user, to him it looks like a standard landing/welcome page. Silver et al. \cite{pw-managment-attacks} claim that six out of ten password managers were vulnerable to this simple attack.


\begin{figure}
    \includegraphics[width=\textwidth]{rouge-wifi-page}
    \caption{Rouge wifi landing page containing iframes with common sites, used to steal password form a autofilling password manager.} 
    \label{rouge-wifi-page}
\end{figure}

\end{example}

\par Even if the password manager is secure against autofill-attacks, or if it does not include the feature, the manager might still be at risk. If the account storing all the usernames and passwords of a user where to be compromised, all the sites would be compromised, so it is important that the password manager is even more secure than the sites themselves. Zhao et al. \cite{lastpass-security} identified several vulnerabilities in the LastPass implementation, even though no known breaches has been reported. They investigate different types of attacks, including attacks on local decrypted credentials; request monitoring attacks which tries to intercept request between the password manager and related cloud-storage; as well as brute-force attacks trying to crack the master password. The conclusion is that password managers are double-edged swords, in theory they help make password authentication stronger, but if implemented slightly wrong may be a major vulnerability. Storing all the passwords at one place makes a obvious point of attack for adversaries since breaking the managers most of the time would break all accounts stored within.


\subsection{Password management schemes}
\todo[inline]{in progress}


\section{Usability Model}
\cite{naturally-rehearsing}


\section{Security Model}

\cite{naturally-rehearsing}











