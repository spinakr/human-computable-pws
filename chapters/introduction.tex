\chapter{Introduction}
\label{chp:intro} 

Passwords have come to be the one authentication method used by nearly all Internet sites and services. Guidelines and policies instructing the users on how their passwords should be, and how they should be maintained are seen a lot. The problem with most of these recommendations, are that they expect too much of the users. The password should be long and complicated, changed every month and not reused on more than one account. These are all commonly seen as recommendations given by sites on the Internet, and it is clear that no user will ever be able to fulfill them without using some kind of scheme or by adapting their passwords to circumvent them. Password reuse or password schemes improvised by the users lead to a major loss in security, which, of course, is the opposite of the intentions when requiring passwords to be long and complex.  
\par It does not seem that passwords as authentication mechanism are falling in popularity. This together with the obvious limitations of the human memory, means that there is a need for strong ways of managing passwords. 


\section{Motivation}
Blocki et al.~\cite{Blocki2014,hcp-blocki} has proposed a scheme allowing human users to calculated passwords based on publicly available challenges and a secret mapping stored in their own memory. This allow users to protect all online accounts using long passwords while only remembering one set of mappings. The proposed scheme relies on generating random challenges for each account, which then have to be visible to users when logging in to the desired account. Blocki et al. propose ideas on how to make it easy for users to both memorize the mappings and do the calculations efficiently. Mnemonic techniques to ease the memorization process and a special layout make the calculations more intuitive.
\par The scheme is designed in such a way that the passwords are safe as long as the secret mappings stay secret. If one password is lost, all the others are unaffected, which is one of the biggest strengths compared with other password management solutions. The security of the scheme solely relies on the secrecy of the mapping, meaning that the \emph{challenges} which are stored, can be lost without passwords being compromised. The passwords are thus \emph{not stored} anywhere, but calculated in the mind of the users. 
\par The usability of a password management scheme is key for it to even be considered by most users. If the requirements for it to function are too high, compared to the security rewards, no user will bother learning it. The usability is mostly related to the time spent calculating the passwords in addition to the failure rate when calculating and initial cost of memorizing a set of mappings.
\par This project investigates how the human computable passwords scheme can be used by real users. Since the scheme is quite complex, it can be useful to have an application taking care of all the required overhead, such as generation and management of challenges. Such an application is designed and implemented, with a goal of making it so that users, without too much introduction, can use it to manage passwords. The application is presented and the usability rewards discussed.
\par Even if the application makes the scheme easy and feasible to use, it might still be too demanding in regards to time spent calculating. The second part of this project investigates how efficient and reliably human users can calculate passwords. An experiment asking the participants to actually calculate passwords based on randomly generated challenges, is designed. The experiment records statistics regarding calculation time and failure rate of the trials. The study is structured as an exploratory experiment, trying to find interesting and possibly important characteristics in the usage statistics recorded through the experiment application.

%\section{Related Work}

\section{Scope and Objectives}
The project first presents human computable passwords as constructed by Blocki~\cite{hcp-blocki}, while also describing other related background material. Throughout this project \emph{human computable passwords} will be referred to as \emph{HCP}, e.g. HCP application, HCP scheme. The objectives of the project are summarized as follows:
\begin{itemize}
    \item Study the scheme with a special focus on the usability parameters, discussing how the different components of the scheme affect the usability.
    \item Design, implement and demonstrate an application realizing the password management scheme.
    \item Discuss if the usability is improved through the chosen design. 
    \item Design, implement and conduct an experiment investigating the limitations imposed by calculation time and failure rates of average users.
    \item Conclude with a hypothesis about the practical limitations or advantages of the password management scheme.
\end{itemize}
\par The experiment is not trying to verify a hypothesis stated in advance, but rather collect data thought to be of interest. Afterwards, the data is presented and analyzed. No concrete result is given, but initial trends and characteristics are discussed.
\paragraph{Limitations.}
It is worth noting that the usability of the scheme is directly based on two things, the calculation performance(including the failure rate) and the effort spent memorizing and rehearsing the secret mapping. The experiment mainly investigates the calculation times and failure rates. The efforts related to the secret mapping are discussed in the background and where it is important, but memorizing a secret mapping is not part of the experiment trials. The reasoning for this decision is that it would be much harder to find participants willing to memorize a set of mappings without somehow compensating them for the efforts.



\section{Outline}
\paragraph{Chapter 2} presents relevant background material, mostly related to passwords and password management. Other methods for storing passwords are presented, showing the difference between password management software and password management schemes. Finally the usability model proposed by Blocki~\cite{Blocki2014} is described.
\paragraph{Chapter 3} describes HCP~\cite{hcp-blocki}, including definitions and human computable functions. Next, some new security features are presented, highlighting the relationship between the security parameters of the scheme, which allows users to adjust the scheme to their needs. A walk-through of the scheme showing how to set it up and how to calculate passwords is given in \autoref{usage}.
\par The author notes that the chapter is partially reference work presenting the scheme of Blocki, but also new thoughts and descriptions (\autoref{sec-params}, \autoref{sec:usability}) highlighting important features of the scheme. 
\paragraph{Chapter 4} presents the HCP Chrome extension. First, the architecture and components of Chrome extensions are described, including security features important to the design. Next, the extension design and implementation are presented, including the building blocks used to realize it. A short introduction to each of the components is given, before the actual implementation is described, with additional explanation of the code given in the appendix. Finally, a demonstration of the application is given, illustrating how it would be used in practice, highlighting the strengths of the application.
\paragraph{Chapter 5} describes the second part of the project, namely the usability experiment. The experiment is conducted through a separate application, which is presented, in addition to the experiment setup and results. This includes important choices and assumptions made trying to mimic the actual user experience when calculating passwords.
\paragraph{Chapter 6} contains concluding remarks summarizing the findings and experiences made throughout the project. Suggestions on further work is also given in this final chapter.


