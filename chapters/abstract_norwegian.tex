\pagestyle{empty}
\begin{abstract}
\par Passordhåndtering er et stort problem i den Internetsentriske verden. Dette prosjektet presenterer passordhånderingssystemet ``human computable passwords'', laget av Blocki et al.~\cite{hcp-blocki}. Systemet gjør det mulig for brukere å kalkulere passord ut ifra offentlig tilgjengelige objekter. Systemet evalueres med hensyn til brukervennlighet, og påvirkende faktorer diskuteres. To forskjellig applikasjoner implementeres, en Google Chrome nettleserutvidelse og en web applikasjon.

\par Chrome-utvidelsen implementerer passordhåndteringssystemet, og drar nytte av styrkene en nettleserutvidelse tilfører. Applikasjonen tar hånd om generering, administrasjon og lagring av objectene som brukes til å kalkulere passord. Google kontoen til brukerene gjør det mulig å lagre informasjon persistent. Smarte funksjonaliteter, muliggjort av Chrome utvidelses rammeverket, gjør det mulig å overvåke sider brukerene besøker. Applikasjonen viser de riktige objektene uten brukermedvirkning.

\par Den andre applikasjonen er et eksperiment lagd som en webapplikasjon og demonstrasjonsside. Passordhåndteringssystemet blir presentert og forklart, før brukere får mulighet til å prøve det. Brukerne blir bedt om å regne ut passord på tid. Kalkuleringstiden og feilraten blir så lagret for hvert forsøk. Dataen blir analysert på en utforskende måte, på utkikk etter interessante egenskaper relatert til brukervenlighet.

\par Eksperimentet viste at feilraten var høy, noe som kan hindre brukervenligheten for noen brukere. Feilraten ble målt til 0.0585, tilsvarende sirka 1 av 17 gale utregninger. Ved å katergorisere brukerkontoer begrenses konsekvensene av den høye feilraten, forskjellige kontoer får forskjellig passordlengde avhengig av sensitivitet.

\par Målet med begge applikasjonene er å utforske om passordhåndteringssystemet kan implementeres på en brukervenlig måte. Og om innsatsen det koster brukeren er liten nok i forhold til sikkerheten systemet tilbyr. Chrome utvidelsen senker terskelen for å begynne å bruke systemet, ved å løse problemer knyttet til administrasjon og lagring. Experimentet konkluderer med at feilraten er en viktig del av brukervenligheten til systemet, og bør utforskes nærmere, da en høy rate kan begrense systemet kraftig.



\end{abstract} 
